\section{Strings}

\begin{algorithm}{\textsc{Knuth-Morris-Pratt}-Algorithmus}
	\begin{methods}
		\method{kmpSearch}{sucht \code{sub} in \code{s}}{\abs{s}+\abs{sub}}
	\end{methods}
	\sourcecode{string/kmp.cpp}
\end{algorithm}

\begin{algorithm}{Z-Algorithmus}
	\begin{methods}[ll]
		$z_i\coloneqq$ Längstes gemeinsames Präfix von $s_0\cdots s_{n-1}$ und $s_i\cdots s_{n-1}$ & \runtime{n}
	\end{methods}
	Suchen: Z-Algorithmus auf \code{P\$S} ausführen, Positionen mit $z_i=\abs{P}$ zurückgeben
	\sourcecode{string/z.cpp}
\end{algorithm}

\begin{algorithm}{Rolling Hash}
	\sourcecode{string/rollingHash2.cpp}
\end{algorithm}

\begin{algorithm}{Pattern Matching mit Wildcards}
	Gegeben zwei strings $A$ und $B$,$B$ enthält $k$ \emph{wildcards} enthält. Sei:
	\begin{align*}
		a_i&=\cos(\alpha_i) + i\sin(\alpha_i) &\text{ mit } \alpha_i&=\frac{2\pi A[i]}{\Sigma}\\
		b_i&=\cos(\beta_i) + i\sin(\beta_i) &\text{ mit } \beta_i&=\begin{cases*}
			\frac{2\pi B[\abs{B}-i-1]}{\Sigma} & falls $B[\abs{B}-i-1]\in\Sigma$ \\
			0 & sonst
		\end{cases*}
	\end{align*}
	$B$ matcht $A$ an stelle $i$ wenn $(b\cdot a)[|B|-1+i]=|B|-k$.
	Benutze FFT um $(b\cdot a)$ zu berechnen.
\end{algorithm}

\begin{algorithm}{\textsc{Manacher}'s Algorithm, Longest Palindrome}
	\begin{methods}
		\method{init}{transformiert \code{string a}}{n}
		\method{manacher}{berechnet Längen der Palindrome in longest}{n}
	\end{methods}
	\sourcecode{string/manacher.cpp}
\end{algorithm}

\begin{algorithm}{Longest Common Subsequence}
	\begin{methods}
		\method{lcss}{findet längste gemeinsame Sequenz}{\abs{a}\*\abs{b}}
	\end{methods}
	\sourcecode{string/longestCommonSubsequence.cpp}
\end{algorithm}

\columnbreak
\begin{algorithm}{\textsc{Aho-Corasick}-Automat}
	\begin{methods}[ll]
		sucht patterns im Text & \runtime{\abs{Text}+\sum\abs{pattern}}
	\end{methods}
	\begin{enumerate}
		\item mit \code{addString(pattern, idx)} Patterns hinzufügen.
        \item rufe \code{buildGraph()} auf
		\item mit \code{state = go(state, idx)} in nächsten Zustand wechseln.
        \item erhöhe dabei \code{dp[state]++}
        \item rufe \code{dfs()} auf. In dp[pattern state] stehen die Anzahl der Matches
	\end{enumerate}
	\sourcecode{string/ahoCorasick.cpp}
\end{algorithm}
\clearpage

\begin{algorithm}{Lyndon und De-Bruijn}
	\begin{itemize}
		\item \textbf{Lyndon-Wort:} Ein Wort das lexikographisch kleiner ist als jede seiner Rotationen.
		\item Jedes Wort kann \emph{eindeutig} in eine nicht ansteigende Folge von Lyndon-Worten zerlegt werden.
		\item Für Lyndon-Worte $u, v$ mit $u<v$ gilt, dass $uv$ auch ein Lyndon-Wort ist.
	\end{itemize}
	\begin{methods}
		\method[, Durchschnitt $\Theta(1)$]{next}{lexikographisch nächstes Lyndon-Wort}{n}
		\method{duval}{zerlegt $s$ in Lyndon-Worte}{n}
		\method{minrotation}{berechnet kleinste Rotation von $s$}{n}
	\end{methods}
	\sourcecode{string/lyndon.cpp}
	\sourcecode{string/duval.cpp}
	\begin{itemize}
		\item \textbf{De-Bruijn-Sequenze $\boldsymbol{B(\Sigma, n)}$:}~~~ein Wort das jedes Wort der Länge $n$ genau einmal als substring enthält (und minimal ist). Wobei $B(\Sigma, n)$ zyklisch betrachtet wird.
		\item es gibt $\frac{(k!)^{k^{n-1}}}{k^{n}}$ verschiedene $B(\Sigma, n)$
		\item $B(\Sigma, n)$ hat Länge $\abs{\Sigma}^n$
	\end{itemize}
	\begin{methods}
		\method{deBruijn}{berechnet ein festes $B(\Sigma, n)$}{\abs{\Sigma}^n}
	\end{methods}
	\sourcecode{string/deBruijn.cpp}
\end{algorithm}

\begin{algorithm}{Suffix-Array}
\begin{methods}
	\method{SuffixArray}{berechnet ein Suffix Array}{\abs{s}\*\log^2(\abs{s})}
	\method{lcp}{berechnet Länge des longest common prefix}{\log(\abs{s})}
	\method{}{von \code{s[x]} und \code{s[y]}}{}
\end{methods}
\sourcecode{string/suffixArray.cpp}
\end{algorithm}

\begin{algorithm}{Suffix-Baum}
	\begin{methods}
		\method{SuffixTree}{berechnet einen Suffixbaum}{\abs{s}}
		\method{extend}{fügt den nächsten Buchstaben aus \code{s} ein}{1}
	\end{methods}
	\sourcecode{string/suffixTree.cpp}
\end{algorithm}

\begin{algorithm}{Suffix-Automaton}
	\sourcecode{string/suffixAutomaton.cpp}
	\begin{itemize}
		\item \textbf{Ist \textit{w} Substring von \textit{s}?}
		Baue Automaten für \textit{s} und wende ihn auf \textit{w} an.
		Wenn alle Übergänge vorhanden sind, ist \textit{w} Substring von \textit{s}.
	
		\item \textbf{Ist \textit{w} Suffix von \textit{s}?}
		Wie oben und prüfe, ob Endzustand ein Terminal ist.
	
		\item \textbf{Anzahl verschiedener Substrings.}
		Jeder Pfad im Automaten entspricht einem Substring.
		Für einen Knoten ist die Anzahl der ausgehenden Pfade gleich der Summe über die Anzahlen der Kindknoten plus 1.
		Der letzte Summand ist der Pfad, der in diesem Knoten endet.
	
		\item \textbf{Wie oft taucht \textit{w} in \textit{s} auf?}
		Sei \textit{p} der Zustand nach Abarbeitung von \textit{w}.
		Lösung ist Anzahl der Pfade, die in \textit{p} starten und in einem Terminal enden.
		Diese Zahl lässt sich wie oben rekursiv berechnen.
		Bei jedem Knoten darf nur dann plus 1 gerechnet werden, wenn es ein Terminal ist.
	\end{itemize}
\end{algorithm}

\begin{algorithm}{Trie}
	\sourcecode{string/trie.cpp}
\end{algorithm}
