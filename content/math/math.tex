\section{Mathe}

\begin{algorithm}{Longest Increasing Subsequence}
	\begin{itemize}
		\item \code{lower\_bound} $\Rightarrow$ streng monoton
		\item \code{upper\_bound} $\Rightarrow$ monoton
	\end{itemize}
	\sourcecode{math/longestIncreasingSubsequence.cpp}
\end{algorithm}
\vfill\null\columnbreak

\begin{algorithm}{Zykel Erkennung}
	\begin{methods}
		\method{cycleDetection}{findet Zyklus von $x_0$ und Länge in $f$}{b+l}
	\end{methods}
	\sourcecode{math/cycleDetection.cpp}
\end{algorithm}

\begin{algorithm}{Permutationen}
	\begin{methods}
		\method{kthperm}{findet $k$-te Permutation \big($k \in [0, n!$)\big)}{n\*\log(n)}
	\end{methods}
	\sourcecode{math/kthperm.cpp}
	\begin{methods}
		\method{permIndex}{bestimmt Index der Permutation \big($\mathit{res} \in [0, n!$)\big)}{n\*\log(n)}
	\end{methods}
	\sourcecode{math/permIndex.cpp}
\end{algorithm}
\clearpage

\subsection{Mod-Exponent und Multiplikation über $\boldsymbol{\mathbb{F}_p}$}
%\vspace{-1.25em}
%\begin{multicols}{2}
\method{mulMod}{berechnet $a \cdot b \bmod n$}{\log(b)}
\sourcecode{math/modMulIterativ.cpp}
%	\vfill\null\columnbreak
\method{powMod}{berechnet $a^b \bmod n$}{\log(b)}
\sourcecode{math/modPowIterativ.cpp}
%\end{multicols}
%\vspace{-2.75em}
\begin{itemize}
	\item für $a > 10^9$ \code{__int128} oder \code{modMul} benutzten!
\end{itemize}

\begin{algorithm}{ggT, kgV, erweiterter euklidischer Algorithmus}
	\runtime{\log(a) + \log(b)}
	\sourcecode{math/extendedEuclid.cpp}
\end{algorithm}

\subsection{Multiplikatives Inverses von $\boldsymbol{x}$ in $\boldsymbol{\mathbb{Z}/m\mathbb{Z}}$}
\textbf{Falls $\boldsymbol{m}$ prim:}\quad $x^{-1} \equiv x^{m-2} \bmod m$

\textbf{Falls $\boldsymbol{\ggT(x, m) = 1}$:}
\begin{itemize}
	\item Erweiterter euklidischer Algorithmus liefert $\alpha$ und $\beta$ mit
	$\alpha x + \beta m = 1$.
	\item Nach Kongruenz gilt $\alpha x + \beta m \equiv \alpha x \equiv 1 \bmod m$.
	\item $x^{-1} :\equiv \alpha \bmod m$
\end{itemize}
\textbf{Sonst $\boldsymbol{\ggT(x, m) > 1}$:}\quad Es existiert kein $x^{-1}$.
% \sourcecode{math/multInv.cpp}
\sourcecode{math/shortModInv.cpp}

\paragraph{Lemma von \textsc{Bézout}}
Sei $(x, y)$ eine Lösung der diophantischen Gleichung $ax + by = d$.
Dann lassen sich wie folgt alle Lösungen berechnen:
\[
\left(x + k\frac{b}{\ggT(a, b)},~y - k\frac{a}{\ggT(a, b)}\right)
\]

\paragraph{\textsc{Pell}-Gleichungen}
Sei $(\overline{x}, \overline{y})$ die Lösung von $x^2 - ny^2 = 1$, die $x>1$ minimiert.
Sei $(\tilde{x}, \tilde{y})$ die Lösung von $x^2-ny^2 = c$, die $x>1$ minimiert. Dann lassen
sich alle Lösungen von $x^2-ny^2=c$ berechnen durch:
\begin{align*}
	x_1&\coloneqq \tilde{x}, & y_1&\coloneqq\tilde{y}\\
	x_{k+1}&\coloneqq \overline{x}x_k+n\overline{y}y_k, & y_{k+1}&\coloneqq\overline{x}y_k+\overline{y}x_k
\end{align*}

\begin{algorithm}{Lineare Kongruenz}
	\begin{itemize}
		\item Kleinste Lösung $x$ für $ax\equiv b\pmod{m}$.
		\item Weitere Lösungen unterscheiden sich um \raisebox{2pt}{$\frac{m}{g}$}, es gibt
		also $g$ Lösungen modulo $m$.
	\end{itemize}
	\sourcecode{math/linearCongruence.cpp}
\end{algorithm}

\begin{algorithm}{Chinesischer Restsatz}
	\begin{itemize}
		\item Extrem anfällig gegen Overflows. Evtl. häufig 128-Bit Integer verwenden.
		\item Direkte Formel für zwei Kongruenzen $x \equiv a \bmod n$, $x \equiv b \bmod m$:
		\[
		x \equiv a - y \cdot n \cdot \frac{a - b}{d} \bmod \frac{mn}{d}
		\qquad \text{mit} \qquad
		d := \ggT(n, m) = yn + zm
		\]
		Formel kann auch für nicht teilerfremde Moduli verwendet werden.
		Sind die Moduli nicht teilerfremd, existiert genau dann eine Lösung,
		wenn $a\equiv~b \bmod \ggT(m, n)$.
		In diesem Fall sind keine Faktoren
		auf der linken Seite erlaubt.
	\end{itemize}
	\sourcecode{math/chineseRemainder.cpp}
\end{algorithm}

\begin{algorithm}{Primzahltest \& Faktorisierung}
	\method{isPrime}{prüft ob Zahl prim ist}{\log(n)^2}
	\sourcecode{math/millerRabin.cpp}
	\method{rho}{findet zufälligen Teiler}{\sqrt[\leftroot{3}\uproot{2}4]{n}}
	\sourcecode{math/rho.cpp}
\end{algorithm}

\begin{algorithm}{Teiler}
	\begin{methods}
		\method{countDivisors}{Zählt Teiler von $n$}{\sqrt[\leftroot{3}\uproot{2}3]{n}}
	\end{methods}
	\sourcecode{math/divisors.cpp}
\end{algorithm}

\begin{algorithm}{Matrix-Exponent}
	\begin{methods}
		\method{precalc}{berechnet $m^{2^b}$ vor}{\log(b)\*n^3}
		\method{calc}{berechnet $m^b\cdot$}{\log(b)\cdot n^2}
	\end{methods}
	\textbf{Tipp:} wenn \code{v[x]=1} und \code{0} sonst, dann ist \code{res[y]} = $m^b_{y,x}$.
	\sourcecode{math/matrixPower.cpp}
\end{algorithm}

\begin{algorithm}{Lineare Rekurrenz}
	\begin{methods}
		\method{BerlekampMassey}{Berechnet eine lineare Rekurrenz $n$-ter Ordnung}{n^2}
		\method{}{aus den ersten $2n$ Werte}{}
	\end{methods}
	\sourcecode{math/berlekampMassey.cpp}
	Sei $f(n)=c_{0}f(n-1)+c_{1}f(n-2)+\dots + c_{n-1}f(0)$ eine lineare Rekurrenz.
	
	\begin{methods}
		\method{kthTerm}{Berechnet $k$-ten Term einer Rekurrenz $n$-ter Ordnung}{\log(k)\cdot \text{mul}(n)}
	\end{methods}
	Die Polynom-Multiplikation kann auch mit NTT gemacht werden!
	\sourcecode{math/linearRecurrence.cpp}
	Alternativ kann der \mbox{$k$-te} Term in \runtime{n^3\log(k)} berechnet werden:
	$$\renewcommand\arraystretch{1.5}
	\setlength\arraycolsep{3pt}
	\begin{pmatrix}
		c_{0} & c_{1} & \smash{\cdots} & \smash{\cdots} & c_{n-1} \\
		1 & 0 & \smash{\cdots} & \smash{\cdots} & 0 \\
		0 & \smash{\ddots} & \smash{\ddots} & & \smash{\vdots} \\
		\smash{\vdots} & \smash{\ddots} & \smash{\ddots} & \smash{\ddots} & \smash{\vdots} \\
		0 & \smash{\cdots} & 0 & 1 & 0 \\
	\end{pmatrix}^k
	\times~~
	\begin{pmatrix}
		f(n-1) \\
		f(n-2) \\
		\smash{\vdots} \\
		\smash{\vdots} \\
		f(0) \\
	\end{pmatrix}
	~~=~~
	\begin{pmatrix}
		f(n-1+k) \\
		f(n-2+k) \\
		\smash{\vdots} \\
		\smash{\vdots} \\
		f(k) \makebox[0pt][l]{\hspace{15pt}$\vcenter{\hbox{\huge$\leftarrow$}}$}\\
	\end{pmatrix}
	$$
\end{algorithm}

\begin{algorithm}{Diskreter Logarithmus}
	\begin{methods}
		\method{solve}{bestimmt Lösung $x$ für $a^x=b \bmod m$}{\sqrt{m}\*\log(m)}
	\end{methods}
	\sourcecode{math/discreteLogarithm.cpp}
\end{algorithm}

\begin{algorithm}{Diskrete Quadratwurzel}
	\begin{methods}
		\method{sqrtMod}{bestimmt Lösung $x$ für $x^2=a \bmod p$ }{\log(p)}
	\end{methods}
	\textbf{Wichtig:} $p$ muss prim sein!
	\sourcecode{math/sqrtModCipolla.cpp}
\end{algorithm}
%\columnbreak

\begin{algorithm}{Primitivwurzeln}
	\begin{itemize}
		\item Primitivwurzel modulo $n$ existiert $\Leftrightarrow$ $n \in \{2,\ 4,\ p^\alpha,\ 2\cdot p^\alpha \mid\ 2 < p \in \mathbb{P},\ \alpha \in \mathbb{N}\}$
		\item es existiert entweder keine oder $\varphi(\varphi(n))$ inkongruente Primitivwurzeln
		\item Sei $g$ Primitivwurzel modulo $n$.
		Dann gilt:\newline
		Das kleinste $k$, sodass $g^k \equiv 1 \bmod n$, ist $k = \varphi(n)$.
	\end{itemize}
	\begin{methods}
		\method{isPrimitive}{prüft ob $g$ eine Primitivwurzel ist}{\log(\varphi(n))\*\log(n)}
		\method{findPrimitive}{findet Primitivwurzel (oder -1)}{\abs{ans}\*\log(\varphi(n))\*\log(n)}
	\end{methods}
	\sourcecode{math/primitiveRoot.cpp}
\end{algorithm}

\begin{algorithm}{Diskrete \textrm{\textit{n}}-te Wurzel}
	\begin{methods}
		\method{root}{bestimmt Lösung $x$ für $x^a=b \bmod m$ }{\sqrt{m}\*\log(m)}
	\end{methods}
	Alle Lösungen haben die Form $g^{c + \frac{i \cdot \phi(n)}{\gcd(a, \phi(n))}}$
	\sourcecode{math/discreteNthRoot.cpp}
\end{algorithm}

\begin{algorithm}{\textsc{Legendre}-Symbol}
	Sei $p \geq 3$ eine Primzahl, $a \in \mathbb{Z}$:
	\vspace{-0.15cm}\begin{align*}
		\hspace*{3cm}\legendre{a}{p} &=
		\begin{cases*}
			\hphantom{-}0 & falls $p~\vert~a$ \\[-1ex]
			\hphantom{-}1 & falls $\exists x \in \mathbb{Z}\backslash p\mathbb{Z} : a \equiv x^2 \bmod p$ \\[-1ex]
			-1 & sonst
		\end{cases*} \\
		\legendre{-1}{p} = (-1)^{\frac{p - 1}{2}} &=
		\begin{cases*}
			\hphantom{-}1 & falls $p \equiv 1 \bmod 4$ \\[-1ex]
			-1 & falls $p \equiv 3 \bmod 4$
		\end{cases*} \\
		\legendre{2}{p} = (-1)^{\frac{p^2 - 1}{8}} &=
		\begin{cases*}
			\hphantom{-}1 & falls $p \equiv \pm 1 \bmod 8$ \\[-1ex]
			-1 & falls $p \equiv \pm 3 \bmod 8$
		\end{cases*}
	\end{align*}
	\begin{align*}
		\legendre{p}{q} \cdot \legendre{q}{p} = (-1)^{\frac{p - 1}{2} \cdot \frac{q - 1}{2}} &&
		\legendre{a}{p} \equiv a^{\frac{p-1}{2}}\bmod p
	\end{align*}
	\vspace{-0.05cm}
	\sourcecode{math/legendre.cpp}
\end{algorithm}

\begin{algorithm}{Lineares Sieb und Multiplikative Funktionen}
	Eine (zahlentheoretische) Funktion $f$ heißt multiplikativ wenn $f(1)=1$ und $f(a\cdot b)=f(a)\cdot f(b)$, falls $\ggT(a,b)=1$.
	
	$\Rightarrow$ Es ist ausreichend $f(p^k)$ für alle primen $p$ und alle $k$ zu kennen.
	
	\begin{methods}
		\method{sieve}{berechnet Primzahlen und co.}{N}
		\method{sieved}{Wert der entsprechenden multiplikativen Funktion}{1}
		
		\method{naive}{Wert der entsprechenden multiplikativen Funktion}{\sqrt{n}}
	\end{methods}
	\textbf{Wichtig:} Sieb rechts ist schneller für \code{isPrime} oder \code{primes}!
	
	\sourcecode{math/linearSieve.cpp}
	\textbf{\textsc{Möbius}-Funktion:}
	\begin{itemize}
		\item $\mu(n)=+1$, falls $n$ quadratfrei ist und gerade viele Primteiler hat
		\item $\mu(n)=-1$, falls $n$ quadratfrei ist und ungerade viele Primteiler hat
		\item $\mu(n)=0$, falls $n$ nicht quadratfrei ist
	\end{itemize}

	\textbf{\textsc{Euler}sche $\boldsymbol{\varphi}$-Funktion:}
	\begin{itemize}
		\item Zählt die relativ primen Zahlen $\leq n$.			
		\item $p$ prim, $k \in \mathbb{N}$:
		$~\varphi(p^k) = p^k - p^{k - 1}$
		
		\item \textbf{Euler's Theorem:}
		Für $b \geq \varphi(c)$ gilt: $a^b \equiv a^{b \bmod \varphi(c) + \varphi(c)} \pmod{c}$. Darüber hinaus gilt: $\gcd(a, c) = 1 \Leftrightarrow a^b \equiv a^{b \bmod \varphi(c)} \pmod{c}$.
		Falls $m$ prim ist, liefert das den \textbf{kleinen Satz von \textsc{Fermat}}:
		$a^{m} \equiv a \pmod{m}$
	\end{itemize}
\end{algorithm}

\begin{algorithm}{Primzahlsieb von \textsc{Eratosthenes}}
	\begin{itemize}
		\item Bis $10^8$ in unter 64MB Speicher (lange Berechnung)
	\end{itemize}
	\begin{methods}
		\method{primeSieve}{berechnet Primzahlen und Anzahl}{N\*\log(\log(N))}
		\method{isPrime}{prüft ob Zahl prim ist}{1}
	\end{methods}
	\sourcecode{math/primeSieve.cpp}
\end{algorithm}

\begin{algorithm}{\textsc{Möbius}-Inversion}
	\begin{itemize}
		\item Seien $f,g : \mathbb{N} \to \mathbb{N}$ und  $g(n) := \sum_{d \vert n}f(d)$.
		Dann ist $f(n) = \sum_{d \vert n}g(d)\mu(\frac{n}{d})$.
		\item $\sum\limits_{d \vert n}\mu(d) =
		\begin{cases*}
			1 & falls $n = 1$\\
			0 & sonst
		\end{cases*}$
	\end{itemize}
	\textbf{Beispiel Inklusion/Exklusion:}
	Gegeben sein eine Sequenz $A={a_1,\ldots,a_n}$ von Zahlen, $1 \leq a_i \leq N$. Zähle die Anzahl der \emph{coprime subsequences}.\newline
	\textbf{Lösung}:
	Für jedes $x$, sei $cnt[x]$ die Anzahl der Vielfachen von $x$ in $A$.
	Es gibt $2^{[x]}-1$ nicht leere Subsequences in $A$, die nur Vielfache von $x$ enthalten.
	Die Anzahl der Subsequences mit $\ggT=1$ ist gegeben durch $\sum_{i = 1}^N \mu(i) \cdot (2^{cnt[i]} - 1)$.
\end{algorithm}

\subsection{LGS über $\boldsymbol{\mathbb{F}_p}$}
\method{gauss}{löst LGS}{n^3}
\sourcecode{math/lgsFp.cpp}

\subsection{LGS über $\boldsymbol{\mathbb{R}}$}
\method{gauss}{löst LGS}{n^3}
\sourcecode{math/gauss.cpp}

\begin{algorithm}{Inversionszahl}
	\vspace{-2pt}
	\sourcecode{math/inversions.cpp}
\end{algorithm}

\begin{algorithm}{Numerisch Extremstelle bestimmen}
	\sourcecode{math/goldenSectionSearch.cpp}
\end{algorithm}

\begin{algorithm}{Numerisch Integrieren, Simpsonregel}
	\sourcecode{math/simpson.cpp}
\end{algorithm}

\begin{algorithm}{Polynome, FFT, NTT \& andere Transformationen}
	Multipliziert Polynome $A$ und $B$.
	\begin{itemize}
		\item $\deg(A \cdot B) = \deg(A) + \deg(B)$
		\item Vektoren \code{a} und \code{b} müssen mindestens Größe
		$\deg(A \cdot B) + 1$ haben.
		Größe muss eine Zweierpotenz sein.
		\item Für ganzzahlige Koeffizienten: \code{(ll)round(real(a[i]))}
		\item \emph{xor}, \emph{or} und \emph{and} Transform funktioniert auch mit \code{double} oder modulo einer Primzahl $p$ falls $p \geq 2^{\texttt{bits}}$
	\end{itemize}
	%\sourcecode{math/fft.cpp}
	%\sourcecode{math/ntt.cpp}
	\sourcecode{math/transforms/fft.cpp}
	\sourcecode{math/transforms/ntt.cpp}
	\sourcecode{math/transforms/bitwiseTransforms.cpp}
	Multiplikation mit 2 transforms statt 3: (nur benutzten wenn nötig!)
	\sourcecode{math/transforms/fftMul.cpp}
\end{algorithm}

\begin{algorithm}{Operations on Formal Power Series}
    \sourcecode{math/transforms/seriesOperations.cpp}
\end{algorithm}

\subsection{Kombinatorik}

\paragraph{Wilsons Theorem}
A number $n$ is prime if and only if
$(n-1)!\equiv -1\bmod{n}$.\\
($n$ is prime if and only if $(m-1)!\cdot(n-m)!\equiv(-1)^m\bmod{n}$ for all $m$ in $\{1,\dots,n\}$)
\begin{align*}
	(n-1)!\equiv\begin{cases}
		-1\bmod{n},&\mathrm{falls}~n \in \mathbb{P}\\
		\hphantom{-}2\bmod{n},&\mathrm{falls}~n = 4\\
		\hphantom{-}0\bmod{n},&\mathrm{sonst}
	\end{cases}
\end{align*}

\paragraph{\textsc{Zeckendorfs} Theorem}
Jede positive natürliche Zahl kann eindeutig als Summe einer oder mehrerer
verschiedener \textsc{Fibonacci}-Zahlen geschrieben werden, sodass keine zwei
aufeinanderfolgenden \textsc{Fibonacci}-Zahlen in der Summe vorkommen.\\
\emph{Lösung:} Greedy, nimm immer die größte \textsc{Fibonacci}-Zahl, die noch
hineinpasst.

\paragraph{\textsc{Lucas}-Theorem}
Ist $p$ prim, $m=\sum_{i=0}^km_ip^i$, $n=\sum_{i=0}^kn_ip^i$ ($p$-adische Darstellung),
so gilt
\vspace{-0.75\baselineskip}
\[
	\binom{m}{n} \equiv \prod_{i=0}^k\binom{m_i}{n_i} \bmod{p}.
\]

%\begin{algorithm}{Binomialkoeffizienten}
\paragraph{Binomialkoeffizienten}
	Die Anzahl der \mbox{$k$-elementigen} Teilmengen einer \mbox{$n$-elementigen} Menge.
	
	\begin{expandtable}
\begin{tabularx}{\linewidth}{|C|}
	\hline
	$\frac{n!}{k!(n - k)!} \hfill=\hfill
	\binom{n}{k} \hfill=\hfill
	\binom{n}{n - k} \hfill=\hfill
	\frac{n}{k}\binom{n - 1}{k - 1} \hfill=\hfill
	\frac{n-k+1}{k}\binom{n}{k - 1} \hfill=\hfill
	\frac{k+1}{n-k}\binom{n}{k + 1} \hfill=\hfill$\\
	
	$\binom{n - 1}{k - 1} + \binom{n - 1}{k} \hfill=\hfill
	\binom{n + 1}{k + 1} - \binom{n}{k + 1} \hfill=\hfill
	(-1)^k \binom{k - n - 1}{k} \hfill\approx\hfill
	2^{n} \cdot \frac{2}{\sqrt{2\pi n}}\cdot\exp\left(-\frac{2(x - \frac{n}{2})^2}{n}\right)$\\
	\grayhline
	
	$\sum\limits_{k = 0}^n \binom{n}{k} = 2^n\hfill
	\sum\limits_{k = 0}^n \binom{k}{m} = \binom{n + 1}{m + 1}\hfill
	\sum\limits_{i = 0}^n \binom{n}{i}^2 = \binom{2n}{n}\hfill
	\sum\limits_{k = 0}^n\binom{r + k}{k} = \binom{r + n + 1}{n}$\\
	
	$\binom{n}{m}\binom{m}{k} = \binom{n}{k}\binom{n - k}{m - k}\hfill
	\sum\limits_{k = 0}^n \binom{r}{k}\binom{s}{n - k} = \binom{r + s}{n}\hfill
	\sum\limits_{i = 1}^n \binom{n}{i} \mathit{Fib}_i = \mathit{Fib}_{2n}$\\
	\hline
\end{tabularx}
\end{expandtable}

	
	\begin{methods}
		\method{precalc}{berechnet $n!$ und $n!^{-1}$ vor}{\mathit{lim}}
		\method{calc\_binom}{berechnet Binomialkoeffizient}{1}
	\end{methods}
	\sourcecode{math/binomial0.cpp}
	Falls $n >= p$ for $\mathit{mod}=p^k$ berechne \textit{fac} und \textit{inv} aber teile $p$ aus $i$ und berechne die häufigkeit von $p$ in $n!$ als $\sum\limits_{i=1}\big\lfloor\frac{n}{p^i}\big\rfloor$

    \begin{methods}
		\method{calc\_binom}{berechnet Binomialkoeffizient $(n \le 61)$}{k}
	\end{methods}
	\sourcecode{math/binomial1.cpp}

    \begin{methods}
		\method{calc\_binom}{berechnet Binomialkoeffizient modulo Primzahl $p$}{p-n}
	\end{methods}
	\sourcecode{math/binomial3.cpp}

%	\begin{methods}
%		\method{calc\_binom}{berechnet Primfaktoren vom Binomialkoeffizient}{n}
%	\end{methods}
%	\textbf{WICHTIG:} braucht alle Primzahlen $\leq n$
%	\sourcecode{math/binomial2.cpp}
%\end{algorithm}

\paragraph{\textsc{Catalan}-Zahlen}
\begin{itemize}
	\item Die \textsc{Catalan}-Zahl $C_n$ gibt an:
	\begin{itemize}
		\item Anzahl der Binärbäume mit $n$ nicht unterscheidbaren Knoten.
		\item Anzahl der validen Klammerausdrücke mit $n$ Klammerpaaren.
		\item Anzahl der korrekten Klammerungen von $n+1$ Faktoren.
		\item Anzahl Möglichkeiten ein konvexes Polygon mit $n + 2$ Ecken zu triangulieren.
		\item Anzahl der monotonen Pfade (zwischen gegenüberliegenden Ecken) in
		einem $n \times n$-Gitter, die nicht die Diagonale kreuzen.
	\end{itemize}
\end{itemize}
\[C_0 = 1\qquad C_n = \sum\limits_{k = 0}^{n - 1} C_kC_{n - 1 - k} =
\frac{1}{n + 1}\binom{2n}{n} = \frac{4n - 2}{n+1} \cdot C_{n-1}\]
\begin{itemize}
	\item Formel $1$ erlaubt Berechnung ohne Division in \runtime{n^2}
	\item Formel $2$ und $3$ erlauben Berechnung in \runtime{n}
\end{itemize}

\paragraph{\textsc{Catalan}-Convolution}
\begin{itemize}
	\item Anzahl an Klammerausdrücken mit $n+k$ Klammerpaaren, die mit $(^k$ beginnen.
\end{itemize}
\[C^k_0 = 1\qquad C^k_n = \sum\limits_{\mathclap{a_0+a_1+\dots+a_k=n}} C_{a_0}C_{a_1}\cdots C_{a_k} =
\frac{k+1}{n+k+1}\binom{2n+k}{n} = \frac{(2n+k-1)\cdot(2n+k)}{n(n+k+1)} \cdot C_{n-1}\]

\paragraph{\textsc{Euler}-Zahlen 1. Ordnung}
Die Anzahl der Permutationen von $\{1, \ldots, n\}$ mit genau $k$ Anstiegen.
Für die $n$-te Zahl gibt es $n$ mögliche Positionen zum Einfügen.
Dabei wird entweder ein Anstieg in zwei gesplitted oder ein Anstieg um $n$ ergänzt.
\[\eulerI{n}{0} = \eulerI{n}{n-1} = 1 \quad
\eulerI{n}{k} = (k+1) \eulerI{n-1}{k} + (n-k) \eulerI{n-1}{k-1}=
\sum_{i=0}^{k} (-1)^i\binom{n+1}{i}(k+1-i)^n\]
\begin{itemize}
	\item Formel $1$ erlaubt Berechnung ohne Division in \runtime{n^2}
	\item Formel $2$ erlaubt Berechnung in \runtime{n\log(n)}
\end{itemize}

\paragraph{\textsc{Euler}-Zahlen 2. Ordnung}
Die Anzahl der Permutationen von $\{1,1, \ldots, n,n\}$ mit genau $k$ Anstiegen.
\[\eulerII{n}{0} = 1 \qquad\eulerII{n}{n} = 0 \qquad\eulerII{n}{k} = (k+1) \eulerII{n-1}{k} + (2n-k-1) \eulerII{n-1}{k-1}\]
\begin{itemize}
	\item Formel erlaubt Berechnung ohne Division in \runtime{n^2}
\end{itemize}

\paragraph{\textsc{Stirling}-Zahlen 1. Ordnung}
Die Anzahl der Permutationen von $\{1, \ldots, n\}$ mit genau $k$ Zyklen.
Es gibt zwei Möglichkeiten für die $n$-te Zahl. Entweder sie bildet einen eigene Zyklus, oder sie kann an jeder Position in jedem Zyklus einsortiert werden.
\[\stirlingI{0}{0} = 1 \qquad
\stirlingI{n}{0} = \stirlingI{0}{n} = 0 \qquad
\stirlingI{n}{k} = \stirlingI{n-1}{k-1} + (n-1) \stirlingI{n-1}{k}\]
\begin{itemize}
	\item Formel erlaubt berechnung ohne Division in \runtime{n^2}
\end{itemize}
\[\sum_{k=0}^{n}\pm\stirlingI{n}{k}x^k=x(x-1)(x-2)\cdots(x-n+1)\]
\begin{itemize}
	\item Berechne Polynom mit FFT und benutzte betrag der Koeffizienten \runtime{n\log(n)^2} (nur ungefähr gleich große Polynome zusammen multiplizieren beginnend mit $x-k$)
\end{itemize}

\paragraph{\textsc{Stirling}-Zahlen 2. Ordnung}
Die Anzahl der Möglichkeiten $n$ Elemente in $k$ nichtleere Teilmengen zu zerlegen.
Es gibt $k$ Möglichkeiten die $n$ in eine $n-1$-Partition einzuordnen.
Dazu kommt der Fall, dass die $n$ in ihrer eigenen Teilmenge (alleine) steht.
\[\stirlingII{n}{1} = \stirlingII{n}{n} = 1 \qquad
\stirlingII{n}{k} = k \stirlingII{n-1}{k} + \stirlingII{n-1}{k-1} =
\frac{1}{k!} \sum\limits_{i=0}^{k} (-1)^{k-i}\binom{k}{i}i^n\]
\begin{itemize}
	\item Formel $1$ erlaubt Berechnung ohne Division in \runtime{n^2}
	\item Formel $2$ erlaubt Berechnung in \runtime{n\log(n)}
\end{itemize}

\paragraph{\textsc{Bell}-Zahlen}
Anzahl der Partitionen von $\{1, \ldots, n\}$.
Wie \textsc{Stirling}-Zahlen 2. Ordnung ohne Limit durch $k$.
\[B_1 = 1 \qquad
B_n = \sum\limits_{k = 0}^{n - 1} B_k\binom{n-1}{k}
= \sum\limits_{k = 0}^{n}\stirlingII{n}{k}\qquad\qquad B_{p^m+n}\equiv m\cdot B_n + B_{n+1} \bmod{p}\]

\paragraph{Partitions}
Die Anzahl der Partitionen von $n$ in genau $k$ positive Summanden.
Die Anzahl der Partitionen von $n$ mit Elementen aus ${1,\dots,k}$.
\begin{align*}
	p_0(0)=1 \qquad p_k(n)&=0 \text{ für } k > n \text{ oder } n \leq 0 \text{ oder } k \leq 0\\
	p_k(n)&= p_k(n-k) + p_{k-1}(n-1)\\[2pt]
	p(n)=\sum_{k=1}^{n} p_k(n)&=p_n(2n)=\sum\limits_{k=1}^\infty(-1)^{k+1}\bigg[p\bigg(n - \frac{k(3k-1)}{2}\bigg) + p\bigg(n - \frac{k(3k+1)}{2}\bigg)\bigg]
\end{align*}
\begin{itemize}
	\item in Formel $3$ kann abgebrochen werden wenn $\frac{k(3k-1)}{2} > n$.
	\item Die Anzahl der Partitionen von $n$ in bis zu $k$ positive Summanden ist $\sum\limits_{i=0}^{k}p_i(n)=p_k(n+k)$.
\end{itemize}

\subsection{The Twelvefold Way \textnormal{(verteile $n$ Bälle auf $k$ Boxen)}}
\begin{expandtable}
\begin{tabularx}{\linewidth}{|C|CICICIC|}
	\hline
	Bälle & identisch & verschieden & identisch      & verschieden \\
	Boxen & identisch & identisch      & verschieden & verschieden \\
	\hline
	-- &
	$p_k(n + k)$ &
	$\sum\limits_{i = 0}^k \stirlingII{n}{i}$ &
	$\binom{n + k - 1}{k - 1}$ &
	$k^n$ \\
	\grayhline

	\makecell{Bälle pro\\Box $\geq 1$} &
	$p_k(n)$ &
	$\stirlingII{n}{k}$ &
	$\binom{n - 1}{k - 1}$ &
	$k! \stirlingII{n}{k}$ \\
	\grayhline

	\makecell{Bälle pro\\Box $\leq 1$} &
	$[n \leq k]$ &
	$[n \leq k]$ &
	$\binom{k}{n}$ &
	$n! \binom{k}{n}$ \\
	\hline
	\multicolumn{5}{|l|}{
		$[\text{Bedingung}]$: \code{return Bedingung ? 1 : 0;}
	} \\
	\hline
\end{tabularx}
\end{expandtable}


\subsection{Platonische Körper}
\begin{expandtable}
\begin{tabularx}{\linewidth}{|X|CCCX|}
	\hline
	Übersicht       & |F|    & |V|   & |E|    & dual zu \\
	\hline
	Tetraeder       & 4      & 4     & 6      & Tetraeder \\
	Würfel          & 6      & 8     & 12     & Oktaeder \\
	Oktaeder        & 8      & 6     & 12     & Würfel\\
	Dodekaeder      & 12     & 20    & 30     & Ikosaeder \\
	Ikosaeder       & 20     & 12    & 30     & Dodekaeder \\
	\hline
	\multicolumn{5}{|c|}{Färbungen mit maximal $n$ Farben (bis auf Isomorphie)} \\
	\hline
	\multicolumn{3}{|l}{|V| vom Oktaeder/|F| vom Würfel} &
	\multicolumn{2}{l|}{$(n^6 + 3n^4 + 12n^3 + 8n^2)/24$} \\

	\multicolumn{3}{|l}{|V| vom Würfel/|F| vom Oktaeder} &
	\multicolumn{2}{l|}{$(n^8 + 17n^4 + 6n^2)/24$} \\

	\multicolumn{3}{|l}{|E| vom Würfel/Oktaeder} &
	\multicolumn{2}{l|}{$(n^{12} + 6n^7 + 3n^6 + 8n^4 + 6n^3)/24$} \\

	\multicolumn{3}{|l}{|V|/|F| vom Tetraeder} &
	\multicolumn{2}{l|}{$(n^4 + 11n^2)/12$} \\

	\multicolumn{3}{|l}{|E| vom Tetraeder} &
	\multicolumn{2}{l|}{$(n^6 + 3n^4 + 8n^2)/12$} \\

	\multicolumn{3}{|l}{|V| vom Ikosaeder/|F| vom Dodekaeder} &
	\multicolumn{2}{l|}{$(n^{12} + 15n^6 + 44n^4)/60$} \\

	\multicolumn{3}{|l}{|V| vom Dodekaeder/|F| vom Ikosaeder} &
	\multicolumn{2}{l|}{$(n^{20} + 15n^{10} + 20n^8 + 24n^4)/60$} \\

	\multicolumn{3}{|l}{|E| vom Dodekaeder/Ikosaeder} &
	\multicolumn{2}{l|}{$(n^{30} + 15n^{16} + 20n^{10} + 24n^6)/60$} \\
	\hline
\end{tabularx}
\end{expandtable}


\begin{tabularx}{\linewidth}{|LICIR|}
	\hline
	\multicolumn{3}{|c|}{
		Wahrscheinlichkeitstheorie ($A,B$ Ereignisse und $X,Y$ Variablen)
	} \\
	\hline
	$\E(X + Y) = \E(X) + \E(Y)$ &
	$\E(\alpha X) = \alpha \E(X)$ &
	$X, Y$ unabh. $\Leftrightarrow \E(XY) = \E(X) \cdot \E(Y)$\\
	
	$\Pr[A \vert B] = \frac{\Pr[A \land B]}{\Pr[B]}$ &
	$A, B$ disj. $\Leftrightarrow \Pr[A \land B] = \Pr[A] \cdot \Pr[B]$ &
	$\Pr[A \lor B] = \Pr[A] + \Pr[B] - \Pr[A \land B]$ \\
	\hline
\end{tabularx}
\vfill
\begin{tabularx}{\linewidth}{|Xlr|lrX|}
	\hline
	\multicolumn{6}{|c|}{\textsc{Bertrand}'s Ballot Theorem (Kandidaten $A$ und $B$, $k \in \mathbb{N}$)} \\
	\hline
	& $\#A > k\#B$ & $Pr = \frac{a - kb}{a + b}$ &
	$\#B - \#A \leq k$ & $Pr = 1 - \frac{a!b!}{(a + k + 1)!(b - k - 1)!}$ & \\
	
	& $\#A \geq k\#B$ & $Pr = \frac{a + 1 - kb}{a + 1}$ &
	$\#A \geq \#B + k$ & $Num = \frac{a - k + 1 - b}{a - k + 1} \binom{a + b - k}{b}$ & \\
	\hline
\end{tabularx}


\subsection{Satz von \textsc{Sprague-Grundy}}
Weise jedem Zustand $X$ wie folgt eine \textsc{Grundy}-Zahl $g\left(X\right)$ zu:
\[
g\left(X\right) := \min\left\{
\mathbb{Z}_0^+ \setminus
\left\{g\left(Y\right) \mid Y \text{ von } X \text{ aus direkt erreichbar}\right\}
\right\}
\]
$X$ ist genau dann gewonnen, wenn $g\left(X\right) > 0$ ist.\\
Wenn man $k$ Spiele in den Zuständen $X_1, \ldots, X_k$ hat, dann ist die \textsc{Grundy}-Zahl des Gesamtzustandes $g\left(X_1\right) \oplus \ldots \oplus g\left(X_k\right)$.

\subsection{Nim-Spiele}
\begin{itemize}
	\item[\ding{182}] letzter gewinnt (normal)
	\item[\ding{183}] letzter verliert
\end{itemize}
\begin{expandtable}
\begin{tabularx}{\linewidth}{|p{0.37\linewidth}|X|}
	\hline
	Beschreibung &
	Strategie \\
	\hline

	$M = [\mathit{pile}_i]$\newline
	$[x] := \{1, \ldots, x\}$&
	$\mathit{SG} = \oplus_{i = 1}^n \mathit{pile}_i$\newline
	\ding{182} Nimm von einem Stapel, sodass $\mathit{SG}$ $0$ wird.\newline
	\ding{183} Genauso.
	Außer: Bleiben nur noch Stapel der Größe $1$, erzeuge ungerade Anzahl solcher Stapel.\\
	\hline

	$M = \{a^m \mid m \geq 0\}$ &
	$a$ ungerade: $\mathit{SG}_n = n \% 2$\newline
	$a$ gerade:\newline
	$\mathit{SG}_n = 2$, falls $n \equiv a \bmod (a + 1) $\newline
	$\mathit{SG}_n = n \% (a + 1) \% 2$, sonst.\\
	\hline

	$M_{\text{\ding{172}}} = \left[\frac{\mathit{pile}_i}{2}\right]$\newline
	$M_{\text{\ding{173}}} =
	\left\{\left\lceil\frac{\mathit{pile}_i}{2}\right\rceil,~
	\mathit{pile}_i\right\}$ &
	\ding{172}
	$\mathit{SG}_{2n} = n$,
	$\mathit{SG}_{2n+1} = \mathit{SG}_n$\newline
	\ding{173}
	$\mathit{SG}_0 = 0$,
	$\mathit{SG}_n = [\log_2 n] + 1$ \\
	\hline

	$M_{\text{\ding{172}}} = \text{Teiler von $\mathit{pile}_i$}$\newline
	$M_{\text{\ding{173}}} = \text{echte Teiler von $\mathit{pile}_i$}$ &
	\ding{172}
	$\mathit{SG}_0 = 0$,
	$\mathit{SG}_n = \mathit{SG}_{\text{\ding{173},n}} + 1$\newline
	\ding{173}
	$\mathit{ST}_1 = 0$,
	$\mathit{SG}_n = \text{\#Nullen am Ende von $n_{bin}$}$\\
	\hline

	$M_{\text{\ding{172}}} = [k]$\newline
	$M_{\text{\ding{173}}} = S$, ($S$ endlich)\newline
	$M_{\text{\ding{174}}} = S \cup \{\mathit{pile}_i\}$ &
	$\mathit{SG}_{\text{\ding{172}}, n} = n \bmod (k + 1)$\newline
	\ding{182} Niederlage bei $\mathit{SG} = 0$\newline
	\ding{183} Niederlage bei $\mathit{SG} = 1$\newline
	$\mathit{SG}_{\text{\ding{174}}, n} = \mathit{SG}_{\text{\ding{173}}, n} + 1$\\
	\hline

	\multicolumn{2}{|l|}{
		Für jedes endliche $M$ ist $\mathit{SG}$ eines Stapels irgendwann periodisch.
	} \\
	\hline

	\textsc{Moore}'s Nim:\newline
	Beliebige Zahl von maximal $k$ Stapeln. &
	\ding{182}
	Schreibe $\mathit{pile}_i$ binär.
	Addiere ohne Übertrag zur Basis $k + 1$.
	Niederlage, falls Ergebnis gleich 0.\newline
	\ding{183}
	Wenn alle Stapel $1$ sind:
	Niederlage, wenn $n \equiv 1 \bmod (k + 1)$.
	Sonst wie in \ding{182}.\\
	\hline

	Staircase Nim:\newline
	$n$ Stapel in einer Reihe.
	Beliebige Zahl von Stapel $i$ nach Stapel $i-1$. &
	Niederlage, wenn Nim der ungeraden Spiele verloren ist:\newline
	$\oplus_{i = 0}^{(n - 1) / 2} \mathit{pile}_{2i + 1} = 0$\\
	\hline

	\textsc{Lasker}'s Nim:\newline
	Zwei mögliche Züge:\newline
	1) Nehme beliebige Zahl.\newline
	2) Teile Stapel in zwei Stapel (ohne Entnahme).&
	$\mathit{SG}_n = n$, falls $n \equiv 1,2 \bmod 4$\newline
	$\mathit{SG}_n = n + 1$, falls $n \equiv 3 \bmod 4$\newline
	$\mathit{SG}_n = n - 1$, falls $n \equiv 0 \bmod 4$\\
	\hline

	\textsc{Kayles}' Nim:\newline
	Zwei mögliche Züge:\newline
	1) Nehme beliebige Zahl.\newline
	2) Teile Stapel in zwei Stapel (mit Entnahme).&
	Berechne $\mathit{SG}_n$ für kleine $n$ rekursiv.\newline
	$n \in [72,83]: \quad 4, 1, 2, 8, 1, 4, 7, 2, 1, 8, 2, 7$\newline
	Periode ab $n = 72$ der Länge $12$.\\
	\hline
\end{tabularx}
\end{expandtable}
	

\subsection{Verschiedenes}
\begin{tabularx}{\linewidth}{|ll|}
	\hline
	\multicolumn{2}{|C|}{Verschiedenes} \\
	\hline
	Türme von Hanoi, minimale Schirttzahl: &
	$T_n = 2^n - 1$ \\

	\#Regionen zwischen $n$ Geraden	&
	$\frac{n\left(n + 1\right)}{2} + 1$ \\

	\#abgeschlossene Regionen zwischen $n$ Geraden &
	$\frac{n^2 - 3n + 2}{2}$ \\

	\#markierte, gewurzelte Bäume	&
	$n^{n-1}$ \\

	\#markierte, nicht gewurzelte Bäume	&
	$n^{n-2}$ \\

	\#Wälder mit $k$ gewurzelten Bäumen	&
	$\frac{k}{n}\binom{n}{k}n^{n-k}$ \\

	\#Wälder mit $k$ gewurzelten Bäumen mit vorgegebenen Wurzelknoten&
	$\frac{k}{n}n^{n-k}$ \\

	Dearangements &
	$!n = (n - 1)(!(n - 1) + !(n - 2)) = \left\lfloor\frac{n!}{e} + \frac{1}{2}\right\rfloor$ \\
	&
	$\lim\limits_{n \to \infty} \frac{!n}{n!} = \frac{1}{e}$ \\
	\hline
\end{tabularx}



\begin{algorithm}{Div Sum}
	\method{divSum}{berechnet $\sum_{i=0}^{n-1} \left\lfloor \frac{a\cdot i + b}{m}  \right\rfloor$}{\log(n)}
	\textbf{Wichtig:} $b$ darf nicht negativ sein!
	\sourcecode{math/divSum.cpp}
\end{algorithm}

\begin{algorithm}{Min Mod}
	\begin{methods}
		\method{firstVal}{berechnet den ersten Wert von $0,\ a, \ldots,\ a \cdot i \bmod{m}$,}{\log(m)}
		\method{}{der in $[l, r]$ liegt. Gibt $-1$ zurück, falls er nicht existiert.}{}
		\method{minMod}{berechnet das Minimum von $(a \cdot i + b) \bmod{m}$ für $i \in [0, n)$}{\log(m)}
	\end{methods}
	\textbf{Wichtig:} $0 \leq a, b, l, r < m$
	\sourcecode{math/minMod.cpp}
\end{algorithm}

\subsection{Reihen}
\begin{expandtable}
\begin{tabularx}{\linewidth}{|XIXIX|}
	\hline
	$\sum\limits_{i = 1}^n i = \frac{n(n+1)}{2}$ &
	$\sum\limits_{i = 1}^n i^2 = \frac{n(n + 1)(2n + 1)}{6}$ &
	$\sum\limits_{i = 1}^n i^3 = \frac{n^2 (n + 1)^2}{4}$ \\
	\grayhline
	
	$\sum\limits_{i = 0}^n c^i = \frac{c^{n + 1} - 1}{c - 1} \hfill c \neq 1$ &	
	$\sum\limits_{i = 0}^\infty c^i = \frac{1}{1 - c} \hfill \vert c \vert < 1$ &
	$\sum\limits_{i = 1}^\infty c^i = \frac{c}{1 - c} \hfill \vert c \vert < 1$ \\
	\grayhline
		
	\multicolumn{2}{|lI}{
		$\sum\limits_{i = 0}^n ic^i = \frac{nc^{n + 2} - (n + 1)c^{n + 1} + c}{(c - 1)^2} \quad c \neq 1$
	} &	
	$\sum\limits_{i = 0}^\infty ic^i = \frac{c}{(1 - c)^2} \hfill \vert c \vert < 1$ \\
	\grayhline	
	
	\multicolumn{2}{|lI}{
		$\sum\limits_{i = 1}^n iH_i = \frac{n(n + 1)}{2}H_n - \frac{n(n - 1)}{4}$
	} &
	$H_n = \sum\limits_{i = 1}^n \frac{1}{i}$ \\
	\grayhline
	
	\multicolumn{2}{|lI}{
		$\sum\limits_{i = 1}^n \binom{i}{m}H_i =
		\binom{n + 1}{m + 1} \left(H_{n + 1} - \frac{1}{m  + 1}\right)$
	} &
	$\sum\limits_{i = 1}^n H_i = (n + 1)H_n - n$ \\
	\hline
\end{tabularx}
\end{expandtable}


\subsection{Wichtige Zahlen}

\begin{tabularx}{\linewidth}{|r||r|r||r|r|r||C|}
	\hline
	\multicolumn{7}{|c|}{Important Numbers} \\
	\hline
	$10^x$ &             Highly Composite &  \# Divs & $<$ Prime & $>$ Prime &                   \# Primes & \\
	\hline
	     1 &                            6 &        4 &      $-3$ &      $+1$ &                           4 & \\
	     2 &                           60 &       12 &      $-3$ &      $+1$ &                          25 & \\
	     3 &                          840 &       32 &      $-3$ &      $+9$ &                         168 & \\
	     4 &                       7\,560 &       64 &     $-27$ &      $+7$ &                      1\,229 & \\
	     5 &                      83\,160 &      128 &      $-9$ &      $+3$ &                      9\,592 & \\
	     6 &                     720\,720 &      240 &     $-17$ &      $+3$ &                     78\,498 & \\
	     7 &                  8\,648\,640 &      448 &      $-9$ &     $+19$ &                    664\,579 & \\
	     8 &                 73\,513\,440 &      768 &     $-11$ &      $+7$ &                 5\,761\,455 & \\
	     9 &                735\,134\,400 &   1\,344 &     $-63$ &      $+7$ &                50\,847\,534 & \\
	    10 &             6\,983\,776\,800 &   2\,304 &     $-33$ &     $+19$ &               455\,052\,511 & \\
	    11 &            97\,772\,875\,200 &   4\,032 &     $-23$ &      $+3$ &            4\,118\,054\,813 & \\
	    12 &           963\,761\,198\,400 &   6\,720 &     $-11$ &     $+39$ &           37\,607\,912\,018 & \\
	    13 &        9\,316\,358\,251\,200 &  10\,752 &     $-29$ &     $+37$ &          346\,065\,536\,839 & \\
	    14 &       97\,821\,761\,637\,600 &  17\,280 &     $-27$ &     $+31$ &       3\,204\,941\,750\,802 & \\
	    15 &      866\,421\,317\,361\,600 &  26\,880 &     $-11$ &     $+37$ &      29\,844\,570\,422\,669 & \\
	    16 &   8\,086\,598\,962\,041\,600 &  41\,472 &     $-63$ &     $+61$ &     279\,238\,341\,033\,925 & \\
	    17 &  74\,801\,040\,398\,884\,800 &  64\,512 &      $-3$ &      $+3$ &  2\,623\,557\,157\,654\,233 & \\
	    18 & 897\,612\,484\,786\,617\,600 & 103\,680 &     $-11$ &      $+3$ & 24\,739\,954\,287\,740\,860 & \\
	\hline
\end{tabularx}


\subsection{Recover $\boldsymbol{x}$ and $\boldsymbol{y}$ from $\boldsymbol{y}$ from $\boldsymbol{x\*y^{-1}}$ }
\method{recover}{findet $x$ und $y$ für $x=x\*y^{-1}\bmod m$}{\log(m)}
\textbf{WICHTIG:} $x$ und $y$ müssen kleiner als $\sqrt{\nicefrac{m}{2}}$ sein!
\sourcecode{math/recover.cpp}

\optional{
\subsection{Primzahlzählfunktion $\boldsymbol{\pi}$}
\begin{methods}
	\method{init}{berechnet $\pi$ bis $N$}{N\*\log(\log(N))}
	\method{phi}{zählt zu $p_i$ teilerfremde Zahlen $\leq n$ für alle $i \leq k$}{???}
	\method{pi}{zählt Primzahlen  $\leq n$ ($n < N^2$)}{n^{2/3}}
\end{methods}
\sourcecode{math/piLehmer.cpp}

\subsection{Primzahlzählfunktion $\boldsymbol{\pi}$}
\sourcecode{math/piLegendre.cpp}
}

\begin{algorithm}[optional]{Big Integers}
	\sourcecode{math/bigint.cpp}
\end{algorithm}
